\chapter{Procesos sobre redes, aprendizaje y ética}

\section*{Introducción general}
Los módulos anteriores presentaron la estructura y mesoestructura de las redes.  Sin embargo, las
redes no son meros objetos estáticos: sobre ellas operan procesos que difunden información,
enfermedades o comportamientos, y a partir de su estructura se pueden extraer representaciones
numéricas útiles para aprendizaje automático.  Además, medir y procesar redes genera
implicaciones éticas y de privacidad.  Este capítulo integra estos tres aspectos: \emph{procesos
dinámicos sobre redes} (Módulo 6), \emph{aprendizaje en grafos} (Módulo 7) y cuestiones de
\emph{ética y privacidad} (Módulo 8).  Cada módulo se presenta con fundamentos
matemáticos, intuiciones y referencias actuales.

\section{Módulo 6 – Procesos sobre redes}

\subsection{Difusión y contagio en redes heterogéneas}

Los modelos clásicos de difusión —como SIS (susceptible–infectado–susceptible) y SIR
(susceptible–infectado–recuperado)— describen el tránsito de estados en nodos de una red.
Consideremos un modelo SIS discreto sobre un grafo $G=(V,E)$ con $|V|=N$ nodos y grados
\(\{k_i\}\).  Cada nodo $i$ puede estar en estado susceptible ($S$) o infectado ($I$).  En cada
paso temporal, una infección se transmite de un nodo infectado $i$ a un vecino susceptible $j$
con probabilidad $\lambda$, y cada nodo infectado se recupera con probabilidad $\mu$.
El número de contactos potenciales depende del grado $k_i$, de modo que la heterogeneidad del
grado influye directamente en la dinámica.  El cociente
\(
R_0 = \frac{\lambda}{\mu}\,\frac{\langle k^2\rangle}{\langle k\rangle}
\)
funciona como un número reproductivo efectivo en redes generalizadas: $\langle k\rangle$ y
$\langle k^2\rangle$ denotan respectivamente la media y la segunda
momento de la distribución de grados\cite{pastorsatorras2015}.  La condición $R_0>1$ marca la
existencia de un umbral epidémico.  En redes \emph{scale–free} con distribución de grado
\(P(k)\propto k^{-\gamma}\) y exponente $2<\gamma\le 3$, el segundo momento diverge,
\(\langle k^2\rangle\to\infty\), de modo que el umbral tiende a cero para $N\to\infty$ y
\(R_0>1\) para cualquier par de valores finitos de $(\lambda,\mu)$【272770960177755†L1790-L1797】.
Esto explica por qué enfermedades o memes se propagan fácilmente en redes muy heterogéneas, y
resalta el papel crucial de los hubs.

\subsection{Influencia de la centralidad en la difusión}

La eficiencia con la que un nodo propaga información no depende únicamente del número de
vecinos, sino de su posición en la estructura global.  Las medidas de centralidad analizadas
anteriormente determinan la capacidad de los nodos para servir como superpropagadores o
barreras.  Rodrigues señala que los nodos centrales son difusores especialmente eficaces y que
la identificación de tales nodos depende fuertemente de la estructura de la red; no existe una
definición universal de centralidad válida para todos los procesos【643362955205334†L69-L82】.
Por ejemplo, en redes con fuerte heterogeneidad, los hubs (alto grado) dominan el contagio;
en redes modulares, los nodos con alta intermediación sirven de puentes entre comunidades.

\subsection{Redes con estructura de mundo pequeño}

El modelo de \emph{Watts–Strogatz}, introducido en el capítulo anterior, tiene implicaciones
décisivas para procesos dinámicos.  En una red regular, las distancias grandes limitan la
propagación; en una red aleatoria, el clustering bajo facilita la difusión pero elimina la
estructura local.  El re‑cableado con probabilidad $p$ preserva el clustering pero crea
\emph{atajos}: unos pocos enlaces largos reducen drásticamente la longitud de camino media
respecto a la red inicial.  Newman sintetiza este efecto: al introducir enlaces aleatorios con
probabilidad pequeña, la longitud media de la red deja de crecer linealmente con el tamaño y
se aproxima rápidamente al régimen logarítmico 【979480736310036†L306-L315】.  En el contexto de
procesos, estos atajos actúan como canales de rápida transmisión, acortando los tiempos de
difusión y explicación de por qué epidemias, rumores o innovaciones se extienden rápidamente en
redes sociales reales.

\subsection{Discusión y perspectivas del módulo 6}

Los procesos de difusión y contagio ilustran la interacción entre topología y dinámica.  Las
redes aleatorias de Erdos–Rényi presentan umbrales epidémicos bien definidos y hacen
predecible la propagación.  Sin embargo, la heterogeneidad y modularidad de las redes reales
introducen fenómenos nuevos: umbrales nulos, hubs superpropagadores y barreras estructurales.
Comprender estos efectos permite diseñar estrategias de vacunación, comunicación o control
adaptadas a la estructura subyacente.

\section{Módulo 7 – Aprendizaje en redes}

\subsection{Representaciones y tareas en grafos}

El aprendizaje automático sobre grafos tiene como objetivo extraer patrones, clasificar nodos o
predicciones de enlaces utilizando la estructura y atributos de la red.  A diferencia del
aprendizaje supervisado convencional, los nodos no son independientes: las etiquetas de
vecinos contienen información relevante (\emph{homofilia}) y patrones de conexión reflejan
relaciones latentes (\emph{equivalencia estructural}).  Hamilton destaca que las tareas
básicas son la clasificación de nodos, la predicción de relaciones (enlaces) y la detección
de comunidades, y que los métodos deben respetar las inductive biases relacionales de los
datos【546120972891789†L846-L900】.

\subsection{Embeddings basados en paseos aleatorios}

Los métodos de incrustación convierten nodos en vectores de un espacio euclidiano de baja
dimensión, preservando relaciones de proximidad.  DeepWalk
(Perozzi et al., 2014) propone recorrer la red mediante paseos aleatorios truncados y tratar
las secuencias de nodos como frases; un modelo de lenguaje tipo skip‑gram aprende
representaciones $\mathbf{z}_i\in\mathbb{R}^d$ que maximizan la probabilidad de co‑ocurrencia
de nodos cercanos en el paseo【527864332286437†L48-L55】.  Formalmente, se maximiza
\(
\sum_{(i,j)\in\mathcal{C}} \log \Pr(i\mid j)\approx \sum_{(i,j)\in\mathcal{C}} \mathbf{z}_i^{\top} \mathbf{z}_j,
\)
donde $\mathcal{C}$ es el conjunto de pares de nodos que aparecen en contextos cercanos.
Este procedimiento captura la proximidad estructural y puede ampliarse mediante sesgos de
paseo para enfatizar homofilia (preferencia por regresar) o búsqueda de roles (exploración
amplia), como en node2vec.

\subsection{Redes neuronales de grafos y message passing}

Una familia de métodos más expresiva son las \emph{redes neuronales de grafos} (GNN), que
aprenden funciones sobre nodos mediante un esquema de \emph{mensaje y actualización}.  Sea
$h_i^{(k)}$ la representación del nodo $i$ en la capa $k$.  Una capa de message passing
realiza dos pasos:

\begin{align}
\text{mensaje:}&\quad m_i^{(k+1)} = \mathop{\mathrm{AGG}}_{j\in\mathcal{N}(i)} \varphi\bigl(h_i^{(k)}, h_j^{(k)}, e_{ij}\bigr),\\
\text{actualización:}&\quad h_i^{(k+1)} = \psi\bigl(h_i^{(k)}, m_i^{(k+1)}\bigr),
\end{align}
\noindent donde $\mathcal{N}(i)$ es el vecindario de $i$, $e_{ij}$ puede codificar atributos de
aristas, y $\varphi$ y $\psi$ son funciones paramétricas (normalmente redes neuronales).  La
agregación debe ser invariante al orden, por ejemplo mediante sumas o promedios.  Al apilar
múltiples capas se obtiene que $h_i^{(K)}$ incorpora información de nodos hasta distancia $K$.
La revisión de Distill subraya que las GNN generalizan las convoluciones al dominar el
agrupamiento de mensajes y permiten explotar la homofilia para tareas como clasificación de
nodos【109701251546256†L472-L512】.

\subsection{Tareas supervisadas y no supervisadas}

\paragraph{Clasificación de nodos.}  En clasificación semisupervisada, dadas etiquetas para un
subconjunto de nodos $V_L\subset V$, se aprende un clasificador $f:V\to\mathcal{Y}$ que
incorpora las representaciones $h_i$ y la estructura de la red.  La pérdida típica combina un
termino de entropía cruzada sobre $V_L$ y regulares con una energía de suavizado que penaliza
grandes diferencias de representaciones entre nodos conectados.

\paragraph{Predicción de enlaces.}  Dada una red incompleta, el objetivo es estimar la
probabilidad de que exista una arista entre $i$ y $j$.  A menudo se utilizan las
representaciones $h_i$ y $h_j$ para calcular una puntuación $\sigma(h_i, h_j)$ y se optimiza
una pérdida binaria sobre enlaces presentes/ausentes.  Este problema es central para
recomendación y detección de nuevas interacciones.

\paragraph{Detección de comunidades.}  Aunque la detección de comunidades se estudia de
manera mesoestructural, las GNN pueden aprender a asignar nodos a grupos latentes cuando se
entrenan para maximizar la modularidad o mediante autoencoders variacionales de grafos.

\subsection{Conclusiones del módulo 7}

El aprendizaje sobre grafos combina principios estadísticos, teoría de redes y aprendizaje
profundo.  Los embeddings basados en paseos proporcionan representaciones sencillas y
eficientes, mientras que las GNN permiten capturar relaciones complejas y atributos
heterogéneos.  Comprender la naturaleza relacional de los datos es esencial para evitar
suposiciones de independencia y explotar la homofilia y la equivalencia estructural.  Las
próximas investigaciones se centran en interpretabilidad, escalabilidad y generalización a
redes dinámicas o heterogéneas.

\section{Módulo 8 – Ética y privacidad en redes}

\subsection{Medir redes como acto de poder}

Registrar interacciones sociales o biológicas y representarlas como redes es un acto que
transforma información local en conocimiento global.  Aaron Clauset argumenta que medir una
red implica tener acceso a información confidencial y, en consecuencia, es un acto de poder
【24689519158805†L19-L38】.  A diferencia de datos tabulares, la interdependencia entre
nodos significa que medir un enlace no sólo revela información sobre sus extremos sino
también sobre su vecindario.  Esta característica amplifica riesgos de privacidad y
confidencialidad.

\subsection{Privacidad, homofilia y filtrado}

La homofilia —tendencia de individuos con atributos similares a conectarse— genera
\emph{fuga de información}: los atributos de un nodo pueden inferirse a partir de la
información de sus vecinos.  Clauset explica que el conocimiento de algunos atributos en
un subconjunto de nodos permite predecir atributos en el resto de la red con alta
exactitud【24689519158805†L109-L126】.  Por ejemplo, en redes sexuales o de salud mental, la
simple observación de enlaces puede revelar patrones sensibles sin el consentimiento de los
individuos.  Los algoritmos de filtrado o recomendación deben tener en cuenta esta
vulnerabilidad para minimizar el riesgo de exposición.

\subsection{Buenas prácticas y principios éticos}

Al trabajar con datos de redes, se deben seguir principios de anonimización y minimización de
daños.  Algunas recomendaciones incluyen:
\begin{itemize}
  \item **Anonimización robusta**: eliminar identificadores directos y aplicar técnicas de
  perturbación que dificulten la reidentificación basada en patrones de conexiones;
  \item **Consentimiento informado**: explicar claramente a los participantes cómo se utilizarán
  sus datos y obtener su autorización explícita;
  \item **Evitar la divulgación de redes sensibles**: abstenerse de publicar redes relacionadas
  con enfermedades, actividad sexual o conductas ilícitas, a menos que se haya evaluado
  cuidadosamente el riesgo y los beneficios sociales;
  \item **Prácticas de equidad**: evaluar algoritmos sobre redes en busca de sesgos y
  desigualdades; corregir desequilibrios que puedan amplificar la marginalización.
\end{itemize}

Aunque no existen recetas universales, combinar las recomendaciones de Clauset con directrices
institucionales (por ejemplo, las propuestas por editoriales académicas y sociedades
científicas) puede mitigar riesgos.  El diseño de sistemas y algoritmos debe incorporar
consideraciones éticas desde su concepción, no sólo como un paso posterior.

\subsection{Conclusiones del módulo 8}

La ética y la privacidad en el análisis de redes no son cuestiones accesoria; la estructura
misma de los datos crea nuevos riesgos y dilemas.  El avance de algoritmos de aprendizaje y
la disponibilidad de datos aumentan la necesidad de marcos normativos que protejan a los
individuos y colectivos.  El investigador en ciencia de redes debe asumir la responsabilidad
de evaluar los impactos sociales de sus métodos y resultados.
