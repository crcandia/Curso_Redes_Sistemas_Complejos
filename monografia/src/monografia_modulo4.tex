\chapter{Modelos generativos en la ciencia de redes}

\section{Introducción}
El análisis estructural de redes reales revela patrones que no son propios de los grafos aleatorios clásicos.  Para comprender el origen de estas regularidades y explicar por qué ciertas propiedades emergen con frecuencia en diferentes dominios (sociales, biológicos o tecnológicos), la ciencia de redes recurre a \emph{modelos generativos}.  Estos modelos formalizan mecanismos de construcción de grafos y permiten reproducir distribuciones de grado, coeficientes de clustering o longitudes de camino medio observados en datos empíricos.  En este módulo presentamos tres paradigmas de generación: las \emph{redes aleatorias} de Erdős–Rényi, el \emph{modelo de re‑cableado de Watts–Strogatz} y el \emph{crecimiento con attachment preferencial} (modelo de Barabási–Albert).  Cada uno captura diferentes características: la aleatoriedad pura explica la aparición de distancias cortas, el re‑cableado introduce clustering sin sacrificar la pequeña longitud media, y el attachment preferencial genera heterogeneidad extrema en forma de distribuciones de grado en ley de potencia.

\section{Redes aleatorias de Erdős–Rényi}
Sea \(G(n,p)\) un grafo construido sobre un conjunto de \(n\) nodos en el que cada una de las \(\tfrac{n(n-1)}{2}\) posibles aristas se incluye de forma independiente con probabilidad \(p\).  Alternativamente, en el modelo \(G(n,M)\) se escogen uniformemente al azar \(M\) aristas entre todas las posibles combinaciones.  En el régimen \(G(n,p)\), la probabilidad de observar un grafo concreto con \(m\) aristas es \(p^m (1-p)^{\binom{n}{2}-m}\).  La \emph{distribución de grados} de un nodo cualquiera en \(G(n,p)\) es binomial,
\begin{equation}
P(k) = \binom{n-1}{k}\,p^k (1-p)^{n-1-k},
\end{equation}
la cual tiende a una distribución de Poisson con media \(\langle k\rangle = p(n-1)\) en el límite de redes grandes y escasas.  Un argumento simple muestra que los grafos aleatorios exhiben el llamado efecto de \emph{mundo pequeño}: si cada nodo tiene un número medio \(z\) de vecinos, entonces el número de nodos a distancia \(d\) de un nodo dado crece como \(z^d\); para alcanzar a todos los nodos basta con que \(z^D \approx n\), lo que implica que la distancia característica crece sólo como \(D \sim \log n / \log z\)【979480736310036†L100-L127】.  Esta relación explica por qué en un grafo con unos pocos cientos de vecinos por nodo se alcanzan millones de nodos en pocas etapas.

Sin embargo, los grafos de Erdős–Rényi carecen de estructura local.  La probabilidad de que dos amigos de un nodo sean amigos entre sí es simplemente la probabilidad \(p\) de que exista una arista al azar, de modo que el coeficiente de clustering global se aproxima a \(C \approx p = \langle k\rangle/(n-1)\), es decir, decrece inversamente con el tamaño de la red【979480736310036†L183-L194】.  Este valor es muy inferior al observado en muchas redes sociales, biológicas y tecnológicas.  Además, la distribución de grados es homogénea: el número de nodos de grado muy alto o muy bajo es exponencialmente improbable.  Estas limitaciones motivaron el desarrollo de modelos alternativos.

\section{Modelo de Watts–Strogatz y fenómeno de mundo pequeño}
El modelo de \emph{Watts–Strogatz} busca generar grafos que combinen las distancias cortas de los grafos aleatorios con el alto clustering de redes ordenadas.  La construcción parte de un anillo con \(n\) nodos en el que cada nodo está conectado a sus \(z\) vecinos más cercanos (\(z\) par para asegurar simetría).  Esta estructura regular presenta un alto coeficiente de clustering y un diámetro grande (crece como \(n/z\)).  Para acortar las distancias se realiza un \emph{re‑cableado}: se revisa cada arista y, con probabilidad \(p\), se elimina uno de sus extremos y se reconecta a un nodo elegido uniformemente al azar en la red【979480736310036†L306-L315】.  Este procedimiento conserva el grado medio \(z\), pero introduce enlaces “largos” que conectan regiones distantes de la red.

Para valores pequeños de \(p\), el coeficiente de clustering \(C\) permanece cercano al de la red regular (\(C\approx 3/4\) para un anillo unidimensional con \(z\gg 2\)), mientras que la longitud de camino medio \(\ell\) desciende abruptamente hasta valores comparables con los de un grafo aleatorio【979480736310036†L329-L347】.  Por ejemplo, para un grafo con \(n=1000\) y \(z=10\), Watts y Strogatz observaron que \(\ell\) se reducía de 50 a 7 con una probabilidad de re‑cableado tan baja como \(p=1/64\)【979480736310036†L329-L347】.  Esta coexistencia de alto clustering y distancias cortas define el fenómeno de \emph{mundo pequeño}.  Variantes posteriores del modelo, como el de Newman–Watts, añaden aristas aleatorias en lugar de re‑cablear, facilitando el análisis analítico del comportamiento de \(\ell\) y la escala característica \(\xi \sim 1/(pz)\)【979480736310036†L306-L315】.

\section{Modelo de Barabási–Albert y attachment preferencial}
Las redes reales a menudo presentan distribuciones de grado con colas pesadas, donde algunos nodos concentran gran número de conexiones (\emph{hubs}).  El modelo de \emph{Barabási–Albert} explica esta heterogeneidad mediante dos ingredientes: \emph{crecimiento} y \emph{attachment preferencial}.  Se inicia con un pequeño grafo de \(m_0\) nodos.  En cada paso se añade un nuevo nodo con \(m\le m_0\) aristas, que se conectan a nodos existentes con probabilidad proporcional a su grado: si el nodo \(i\) tiene grado \(k_i\), entonces la probabilidad de que el nuevo nodo se conecte a \(i\) es
\begin{equation}
\Pi(k_i) = \frac{k_i}{\sum_j k_j}.
\end{equation}
Esta regla de “los ricos se hacen más ricos” implica que los nodos altamente conectados atraen más enlaces conforme la red crece【928400721627140†L230-L237】.  Si se eliminaran la regla de crecimiento o el attachment preferencial y se conectaran los nuevos nodos de manera uniforme, la distribución de grados resultante decaería exponencialmente【928400721627140†L264-L266】.

El análisis de Barabási y Albert muestra que este proceso genera una distribución de grados en ley de potencia.  Sea \(P(k,t)\) la fracción de nodos con grado \(k\) en el instante \(t\); al resolver la ecuación maestra para \(P(k,t)\) se obtiene, en el límite estacionario, una distribución estacionaria
\begin{equation}
P(k) = \frac{2m^2}{k^3},
\end{equation}
lo que corresponde a un exponente de grado \(\gamma = 3\), independiente de \(m\)【928400721627140†L288-L315】.  Esta ley de potencia reproduce la presencia de hubs observada en muchas redes.  El modelo también predice que el grado de un nodo crece en el tiempo como \(k_i(t) \propto t^{1/2}\), de modo que los nodos más antiguos tienden a ser más conectados.  Extensiones del modelo incluyen attachment no lineal (\(\Pi(k) \propto k^\alpha\)) y la incorporación de “fitness” para cada nodo, que modulan la velocidad con la que ganan conexiones.

\section{Discusión y perspectivas}
Los modelos generativos permiten vincular mecanismos simples con propiedades estructurales complejas.  El modelo de Erdős–Rényi captura la aparición de distancias cortas y transitividad nula, pero falla en reproducir el clustering y la heterogeneidad de grado observados en redes reales.  El modelo de Watts–Strogatz introduce enlaces largos sobre una estructura ordenada, produciendo simultáneamente alto clustering y pequeñas distancias.  Finalmente, el modelo de Barabási–Albert explica la emergencia de distribuciones de grado en ley de potencia a partir de reglas de crecimiento y preferencia.  Aun así, estos modelos son simplificaciones: las redes reales pueden presentar modularidad, correlaciones de grado, pesos heterogéneos o dinámicas temporales que requieren modelos más elaborados.  Combinaciones de los mecanismos descritos, como el attachment preferencial con re‑cableado o el crecimiento dependiente de la topología local, siguen siendo objeto de investigación.

\newpage
\bibliographystyle{plainnat}
\bibliography{monografia_modulo4}