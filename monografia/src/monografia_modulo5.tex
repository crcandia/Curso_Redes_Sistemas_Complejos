% -*- coding: utf-8 -*-
\chapter{Organización mesoestructural}

\section{Introducción}
El análisis de redes no sólo se interesa en propiedades de nodos individuales (micro) o en características globales del grafo (macro).  En un nivel intermedio, denominado \emph{mesoestructural}, aparecen patrones de organización que reflejan la interacción de conjuntos de nodos.  Dos ejemplos paradigmáticos de estas estructuras son las \emph{comunidades}, es decir, grupos de nodos con muchas conexiones internas y pocas conexiones externas, y las \emph{estructuras núcleo–periferia}, donde existe un conjunto central densamente conectado y un conjunto periférico con pocas conexiones entre sí.  Otro concepto relacionado es el de \emph{roles estructurales}, que agrupa nodos según patrones de conectividad similares en lugar de densidades de enlace.  Comprender y cuantificar estas mesoestructuras es fundamental para explicar fenómenos emergentes en redes sociales, biológicas, tecnológicas y económicas.

\section{Detección de comunidades}
En una red no dirigida $G=(V,E)$ con $|V|=N$ nodos y $|E|=m$ aristas, una comunidad o \emph{módulo} es un subconjunto $C\subseteq V$ de nodos que posee una alta densidad de aristas internas en comparación con las aristas que lo conectan con el resto de la red.  La detección de comunidades es crucial para descomponer el grafo en unidades funcionales que suelen corresponder a grupos de actores con intereses comunes, módulos funcionales en redes biológicas o tópicos en redes de co-ocurrencia.  \citet{fortunato2010} muestran que la identificación de comunidades es esencial para interpretar la organización de sistemas reales y proporcionan una revisión exhaustiva de métodos basados en la densidad, el flujo y la dinámica de procesos sobre redes.

Existen numerosos algoritmos para la detección de comunidades: desde enfoques divisivos basados en la eliminación de aristas con alta intermediación (algoritmo de Girvan--Newman) hasta métodos jerárquicos y heurísticos que maximizan funciones de calidad.  A continuación se presenta la función de calidad más empleada, la \emph{modularidad}.

\section{Modularidad y su maximización}
La \emph{modularidad} $Q$ es una métrica que cuantifica la diferencia entre la fracción de aristas que caen dentro de las comunidades y la fracción esperada bajo un modelo nulo que conserva el grado de cada nodo.  Consideremos una partición de $V$ en comunidades $\{C_1,\dots,C_k\}$ e introduzcamos la matriz de adyacencia $A\in\{0,1\}^{N\times N}$ y los grados $k_i=\sum_j A_{ij}$.  El modelo nulo más habitual es el modelo de \emph{configuración}, en el que el número de aristas entre $i$ y $j$ se modela como $\tfrac{k_i k_j}{2m}$.  La modularidad se define como \citep{fortunato2010}
\begin{equation}
  Q = \frac{1}{2m} \sum_{i,j} \bigl(A_{ij} - \tfrac{k_i k_j}{2m}\bigr) \delta(C_i, C_j),
  \label{eq:modularity}
\end{equation}
donde $\delta(C_i,C_j)$ es igual a 1 si los nodos $i$ y $j$ pertenecen a la misma comunidad y 0 en caso contrario.  Este valor se puede reescribir en términos de propiedades de cada comunidad $c$ como
\begin{equation}
  Q = \sum_{c} \Bigl[ \frac{l_c}{m} - \Bigl( \frac{d_c}{2m} \Bigr)^2 \Bigr],
  \label{eq:modularity-community}
\end{equation}
donde $l_c=\sum_{i,j\in C_c} A_{ij}/2$ es el número de aristas internas de la comunidad $c$ y $d_c=\sum_{i\in C_c} k_i$ es la suma de los grados de sus nodos.  La modularidad valora positivamente las comunidades que contienen más aristas de las esperadas al azar y penaliza aquellas que sólo reflejan la distribución de grados global.

\paragraph{Interpretación y limitaciones.}
La modularidad fue popularizada por \citet{newman2006} como criterio para detectar comunidades maximizando $Q$.  Según el planteamiento de \citet{viles2023}, la modularidad “maximiza la diferencia entre la fracción de aristas dentro de las comunidades y la fracción esperada si las aristas se asignaran aleatoriamente conservando la distribución de grados”.  Su simplicidad ha motivado algoritmos eficientes como el método de Louvain y sus variantes.  Sin embargo, la modularidad presenta la \emph{resolución límite}: puede no detectar comunidades muy pequeñas en redes grandes, ya que la contribución de una comunidad al valor total $Q$ disminuye al crecer el tamaño de la red.  Además, existen particiones degeneradas con valores similares de modularidad, lo que dificulta la interpretación única de los resultados.

\section{Estructura núcleo–periferia}
Otra mesoestructura relevante es la \emph{estructura núcleo–periferia} (\emph{core–periphery}).  A diferencia de las comunidades, que buscan conjuntos internamente densos y mutuamente poco conectados, la estructura núcleo–periferia divide los nodos en dos clases: el \emph{núcleo} y la \emph{periferia}.  Según \citet{yanchenko2023}, en una red con estructura núcleo–periferia “la asignación de los nodos consta de dos grupos, un núcleo y una periferia.  Los nodos del núcleo están densamente conectados entre sí y además se conectan frecuentemente con los nodos periféricos, mientras que los nodos periféricos tienen pocas conexiones entre ellos”.  Otra forma de caracterizar el núcleo es que sus nodos están a corta distancia de todos los demás nodos.

Esta organización surge, por ejemplo, en redes de comercio mundial: los países con economías grandes (núcleo) comercian entre sí y con economías pequeñas, mientras que las economías pequeñas (periferia) comercian poco entre ellas.  Un patrón similar se observa en redes de aeropuertos, donde los grandes aeropuertos tienen conexiones con otros hubs y con aeropuertos regionales, mientras que los aeropuertos regionales tienen pocos vuelos entre sí.  La presencia de una estructura núcleo–periferia influye en la difusión de información y en la resiliencia del sistema: los nodos centrales suelen ser más influyentes o críticos para la conectividad global \citep{yanchenko2023}.

\subsection*{Modelos y detección}
Existen diversos enfoques para detectar estructuras núcleo–periferia.  Algunos métodos combinan estimación estadística con modelos de bloques, en los que se asume que la matriz de adyacencia puede aproximarse mediante bloques densos (núcleo) y bloques dispersos (periferia).  Otros métodos emplean técnicas espectrales basadas en autovectores de la matriz de adyacencia o del Laplaciano.  La detección de núcleo–periferia comparte similitudes conceptuales con el problema de \emph{rich-club}, donde se estudia la sobreconectividad entre los nodos de grado elevado.

\section{Roles estructurales}
Las comunidades y las estructuras núcleo–periferia describen cómo los nodos se agrupan según la densidad de sus conexiones.  Por el contrario, el concepto de \emph{rol} agrupa nodos según su patrón de conexiones, independientemente del número total de aristas que comparten entre sí.  Siguiendo a \citet{rossi2014}, el descubrimiento de roles fue definido inicialmente como “cualquier proceso que divide los nodos de un grafo en clases de nodos \emph{estructuralmente equivalentes}”.  Dos nodos son estructuralmente equivalentes si mantienen las mismas conexiones con el resto de la red.  En la práctica, esta definición estricta se relaja buscando patrones de conectividad similares.

\paragraph{Equivalencias y tipos de roles.}
Los roles pueden formalizarse mediante tres nociones de equivalencia \citep{rossi2014}:
\begin{enumerate}
  \item \emph{Equivalencia estructural}: dos nodos $i$ y $j$ son estructuralmente equivalentes si $A_{i\ast}=A_{j\ast}$, es decir, tienen exactamente los mismos vecinos.  Esta equivalencia es muy estricta y rara en redes grandes.
  \item \emph{Equivalencia regular}: dos nodos son regularmente equivalentes si sus vecinos pertenecen a las mismas clases de equivalencia.  Permite agrupar, por ejemplo, a actores que ejercen roles de “intermediario” aunque conecten con diferentes individuos.
  \item \emph{Equivalencia estocástica}: dos nodos son estocásticamente equivalentes si las probabilidades de conectarse a otros nodos dependen únicamente de sus clases, lo cual es la base del modelo de bloques estocásticos.
\end{enumerate}
Los roles capturan patrones como nodos estrella (centros de estrella), nodos periféricos, nodos puente o nodos casi clique, entre otros.  A diferencia de las comunidades, los roles no necesitan conformar subconjuntos densos, sino que reflejan funciones o posiciones relativas en la red.  Métodos recientes proponen descubrir roles a partir de representaciones de atributos o de incrustaciones (embeddings) aprendidas sobre la red \citep{rossi2014}.

\section{Discusión y perspectivas}
La organización mesoestructural de una red tiene implicancias profundas en su dinámica y en la interpretación de sus datos.  Las comunidades permiten segmentar la red en grupos cohesionados, la estructura núcleo–periferia revela jerarquías de conectividad e influencia, y los roles estructurales identifican patrones funcionales que no dependen de la densidad de enlaces.  La modularidad ha sido un criterio popular para detectar comunidades, pero presenta limitaciones como la resolución y la existencia de máximos locales; se han desarrollado alternativas y extensiones para mitigar estos problemas.  Las estructuras núcleo–periferia son complementarias a las comunidades y requieren métodos específicos basados en modelos de bloques o técnicas espectrales.  Por su parte, la detección de roles abre un campo distinto enfocado en la equivalencia y la similitud estructural, con aplicaciones en identificación de anomalías y análisis funcional.

En redes reales es común encontrar la coexistencia de varias mesoestructuras: comunidades que exhiben subestructuras núcleo–periferia o nodos que pertenecen a un mismo rol dentro de diferentes comunidades.  El estudio de estas interacciones sigue siendo un área activa de investigación.  Además, incorporar información exógena y considerar restricciones demográficas, funcionales o de equidad en la identificación de comunidades —como propone \citet{viles2023}— es fundamental para análisis aplicados más robustos y responsables.

% Bibliografía: las referencias se cargan desde el archivo .bib correspondiente
\newpage
\nocite{fortunato2010,viles2023,barabasi2016,yanchenko2023,rossi2014}