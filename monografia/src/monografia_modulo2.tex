% Monografía para el Módulo 2 del curso introductorio de Ciencia de Redes.
% En este módulo se abordan las nociones estructurales básicas.
% Este documento usa referencias en formato BibTeX incluidas en
% ``monografia_modulo2.bib``. Compile con pdflatex y biber (o BibTeX).
\documentclass[11pt]{article}

% Codificación y lengua
\usepackage[utf8]{inputenc}
\usepackage[T1]{fontenc}
\usepackage[spanish]{babel}
\usepackage{csquotes}

% Paquetes matemáticos y símbolos
\usepackage{amsmath, amssymb}
\usepackage{bm}

% Paquete para citas con autor–año
\usepackage{natbib}
\bibliographystyle{apalike}

% Paquete para tablas y gráficos si se necesitan
\usepackage{graphicx}
\usepackage{booktabs}

% Ajustes de márgenes
\usepackage[a4paper,margin=2.5cm]{geometry}

\title{Módulo 2: Estructura básica de las redes}
\author{}
\date{}

\begin{document}

\maketitle

\section{Introducción}

Tras haber presentado la noción general de red como modelo de sistemas complejos en el primer módulo, el siguiente paso es comprender su estructura interna.  La estructura de una red determina cómo se propaga la información, cómo responden los sistemas a perturbaciones y qué tan robustos son frente a fallos o ataques.  En este módulo abordamos las herramientas básicas para describir y cuantificar la estructura: distribución de grados, matrices de adyacencia, redes ponderadas, grafos bipartitos y coeficientes de clustering.  También introducimos las principales medidas de centralidad, que permiten identificar la importancia relativa de los nodos y los enlaces.

\section{Distribución de grados}

Sea $G=(V,E)$ un grafo no dirigido con $n=|V|$ nodos.  El \emph{grado} de un nodo $i$ se define como en la ecuación\nbref{eq:degree}, es decir, $k_i=\sum_{j=1}^n A_{ij}$, donde $A$ es la matriz de adyacencia.  La \emph{distribución de grados} $P(k)$ es la probabilidad de que un nodo elegido al azar tenga grado $k$.  En términos discretos, si $N_k$ es el número de nodos con grado $k$, entonces $P(k)=N_k/n$, con $\sum_k P(k) = 1$\citep{barabasi2016}.  Esta distribución constituye un resumen de la heterogeneidad del grafo: redes homogéneas tienen distribuciones concentradas alrededor de un valor medio, mientras que redes heterogéneas exhiben colas pesadas, con hubs de alto grado.  La forma de $P(k)$ influye en la conectividad global, la robustez y la propagación de procesos en la red【70724115848529†L574-L591】.

La distribución de grados es relevante, por ejemplo, para comprender por qué algunas redes reales son robustas a fallos aleatorios pero vulnerables a ataques dirigidos.  En redes de \emph{escala libre} con $P(k)\propto k^{-\gamma}$ y $\gamma\in(2,3)$, la presencia de hubs hace que la eliminación aleatoria de nodos raramente afecte a la conectividad, pero un ataque selectivo a los nodos de mayor grado puede fragmentar la red.

\section{Matriz de adyacencia y grafos dirigidos}

La descripción completa de un grafo puede codificarse en su \emph{matriz de adyacencia} $A\in\mathbb{R}^{n\times n}$.  Para un grafo no dirigido y no ponderado se define
\begin{equation}
  A_{ij}=
  \begin{cases}
    1,&\text{si existe una arista entre $i$ y $j$},\\
    0,&\text{en caso contrario},
  \end{cases}
  \label{eq:adj_matrix}
\end{equation}
de modo que $A$ es simétrica.  En grafos dirigidos se utiliza una convención similar, pero el hecho de que la arista vaya de $i$ a $j$ se refleja en la entrada $A_{ij}$; por tanto $A$ puede no ser simétrica.  La matriz de adyacencia permite expresar muchas propiedades locales y globales de la red de forma algebraica.  Por ejemplo, el grado de salida de un nodo $i$ en un grafo dirigido es $k_i^{\mathrm{out}}=\sum_{j}A_{ij}$ y el grado de entrada es $k_i^{\mathrm{in}}=\sum_{j}A_{ji}$【70724115848529†L757-L777】.

El uso de $A$ facilita el cómputo de caminos de longitud $l$: la entrada $(A^l)_{ij}$ cuenta el número de caminos de longitud $l$ que conectan $i$ con $j$.  Además, muchas medidas de centralidad se definen a partir de funciones de $A$, como la centralidad eigenvectorial (véase la Sección\nbref{sec:centralidades}).

\section{Redes ponderadas}

En numerosas aplicaciones los enlaces llevan asociada una intensidad o peso $w_{ij}$ que cuantifica la fortaleza de la relación (p.ej., número de correos entre dos personas, capacidad de una línea de transmisión).  Un \emph{grafo ponderado} se representa mediante una matriz $W$ cuyas entradas $W_{ij}$ son los pesos.  Para redes ponderadas, el grado generalizado de un nodo puede definirse como la \emph{fuerza} $s_i=\sum_{j}W_{ij}$, que mide la suma de los pesos de las aristas incidentes.  La decisión de ignorar o no los pesos tiene consecuencias analíticas: simplificar una red ponderada a un grafo no ponderado pierde información sobre la intensidad de las interacciones.  \citet{barabasi2016} advierten que muchas redes reales son inherentemente ponderadas y que el análisis puramente topológico puede ser insuficiente【70724115848529†L1158-L1173】.

\section{Redes bipartitas}

Un \emph{grafo bipartito} es una red cuyos nodos pueden dividirse en dos conjuntos disjuntos $U$ y $V$ de modo que cada arista conecta un nodo de $U$ con uno de $V$, y no existen aristas entre nodos del mismo conjunto.  Ejemplos típicos son las redes de colaboración autor–artículo y las redes de recomendación, donde $U$ representa usuarios y $V$ objetos.  Una característica importante es que los grafos bipartitos permiten construir \emph{proyecciones} sobre cada conjunto: por ejemplo, dos autores están conectados en la proyección si han coescrito al menos un artículo.  Sin embargo, estas proyecciones pueden introducir sesgos; nodos con alta actividad en el bipartito generan conexiones spuriamente altas en la proyección.  \citet{barabasi2016} presentan varios ejemplos de redes bipartitas, como el grafo de actores de Hollywood y la red de enfermedades humanas y genes【70724115848529†L1254-L1276】.

\section{Coeficiente de clustering y estructura local}

El \emph{coeficiente de clustering} mide la propensión de los vecinos de un nodo a estar conectados entre sí, es decir, la densidad de triángulos alrededor de un nodo.  Para un nodo $i$ de grado $k_i\geq 2$, el coeficiente de clustering local se define como
\begin{equation}
  C_i = \frac{2 T_i}{k_i(k_i-1)},
  \label{eq:local_clustering}
\end{equation}
donde $T_i$ es el número de triángulos (tripletas completas) que incluyen a $i$【70724115848529†L2317-L2345】.  El factor 2 contabiliza cada triángulo una sola vez.  El coeficiente de clustering medio de la red es $\langle C\rangle=\frac{1}{n}\sum_{i=1}^n C_i$.  En grafos aleatorios clásicos, $\langle C\rangle$ tiende a cero cuando $n\to\infty$, mientras que en redes sociales reales suele ser alto, reflejando la regla informal de ``los amigos de mis amigos tienden a ser amigos''.  La coexistencia de altos coeficientes de clustering con pequeñas distancias promedio se denomina \emph{fenómeno de mundo pequeño}.  El modelo de Watts–Strogatz muestra cómo una pequeña probabilidad de re‑cableado $\beta$ en una red regular reduce drásticamente la longitud de camino medio mientras mantiene el clustering alto【603183041358703†L322-L364】.

\section{Caminos, distancias y pequeña‐mundo}

La \emph{distancia} entre dos nodos $i$ y $j$ es la longitud del camino más corto (número de aristas) que los conecta; se denota $d(i,j)$.  El \emph{diámetro} de una red es la distancia máxima entre pares de nodos conectados y la \emph{longitud de camino media} es $\langle d\rangle = \frac{1}{n(n-1)} \sum_{i\ne j} d(i,j)$.  En muchas redes reales se observa que $\langle d\rangle$ crece lentamente con el tamaño del grafo (de forma logarítmica o sublogarítmica), lo que se denomina la propiedad de mundo pequeño.  La coexistencia de un alto clustering con distancias cortas fue formalizada por Watts y Strogatz en su modelo, en el que se parte de un anillo regular y se re‑cablean aristas con probabilidad $\beta$【603183041358703†L322-L364】.  Para $\beta$ pequeño, el clustering se mantiene cercano al del anillo regular pero la longitud de camino media disminuye rápidamente, explicando la paradoja de las ``seis personas de distancia''.

\section{Centralidades y medidas de importancia}
\label{sec:centralidades}

En el análisis de redes es crucial identificar nodos o aristas importantes desde el punto de vista estructural o funcional.  Las \emph{medidas de centralidad} cuantifican diferentes nociones de importancia.  Presentamos brevemente las más usadas y su interpretación.

\subsection{Centralidad de grado}

La forma más simple de centralidad es el grado $k_i$ ya definido: nodos con alto grado tienen muchas conexiones y, en contextos sociales, pueden interpretarse como individuos populares.  Su principal limitación es que no considera la calidad ni la posición de las conexiones.  \citet{rodrigues2019} destacan que aunque el grado es informativo, no existe una única medida de importancia válida para todas las redes【643362955205334†L66-L83】.

\subsection{Centralidad de intermediación}

La \emph{centralidad de intermediación} $C_B(i)$ se define en la ecuación\nbref{eq:betweenness}.  Un nodo con alta intermediación actúa como puente en muchos caminos geodésicos, por lo que su eliminación puede desconectar la red o retrasar la difusión.  El cálculo exacto de $C_B(i)$ tiene un costo computacional alto para grafos grandes, lo que motiva variantes basadas en muestreo o paseos aleatorios【643362955205334†L295-L359】.

\subsection{Centralidad eigenvectorial}

La \emph{centralidad eigenvectorial} asigna un valor alto a los nodos que están conectados a otros nodos importantes.  Se define como el vector propio principal de la matriz de adyacencia: sea $\bm{v}$ un vector no nulo y $\lambda$ un escalar tales que
\begin{equation}
  A\,\bm{v} = \lambda \, \bm{v},
  \label{eq:eigenvector}
\end{equation}
con $\lambda$ el mayor valor propio de $A$.  La componente $v_i$ de $\bm{v}$ da la centralidad eigenvectorial del nodo $i$.  Intuitivamente, un nodo es importante si se conecta con otros nodos importantes.  Esta medida está relacionada con el número de caminos de todas las longitudes que comienzan en un nodo【643362955205334†L295-L359】.

\subsection{Centralidad de cercanía}

Otra medida clásica es la \emph{centralidad de cercanía}, definida como el inverso de la suma de distancias desde $i$ a todos los demás nodos:
\begin{equation}
  C_C(i) = \frac{n-1}{\sum_{j\ne i} d(i,j)}.
  \label{eq:closeness}
\end{equation}
Esta centralidad identifica nodos que están, en promedio, a poca distancia del resto de la red.  Es útil para encontrar actores que pueden difundir información rápidamente o recolectar recursos de manera eficiente.

Ninguna de estas centralidades es universalmente la mejor.  Su relevancia depende del contexto y del proceso de interés; a menudo conviene considerar varias en conjunto y analizar su correlación con observables empíricos【643362955205334†L66-L83】.

\section{Conclusiones}

En este módulo se han introducido las herramientas básicas para describir la estructura de una red.  Hemos definido la distribución de grados y discutido su impacto en la heterogeneidad y la robustez de las redes.  La matriz de adyacencia proporciona una representación algebraica compacta que permite computar grados y caminos, y su extensión a grafos ponderados resalta la importancia de los pesos en aplicaciones reales.  Los grafos bipartitos constituyen un caso especial que exige cautela al proyectarlos.  El coeficiente de clustering y la distancia media capturan propiedades locales y globales cuya combinación da lugar al fenómeno de mundo pequeño.  Finalmente, las medidas de centralidad—grado, intermediación, eigenvectorial y cercanía—permiten evaluar la importancia estructural de nodos y aristas, aunque ninguna es definitiva para todas las redes.  Estas nociones serán fundamentales para los módulos posteriores dedicados a modelos generativos y dinámicas, así como para el aprendizaje sobre grafos.

% Fin del documento.  Las referencias se incluyen a través de BibTeX.
\newpage
\bibliography{monografia_modulo2}

\end{document}